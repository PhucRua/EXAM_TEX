\documentclass{article}
\usepackage{amsmath, amssymb}
\usepackage{tikz}
\usepackage{tkz-tab}

\begin{document}

\title{Đề thi Toán học lớp 12}
\author{Bộ môn Toán}
\date{Năm học 2024-2025}

\maketitle

\section{Câu trắc nghiệm}

\begin{ex}
	Gieo ngẫu nhiên $2$ đồng tiền thì không gian mẫu của phép thử có bao nhiêu phần tử?
	\choice
	{\True $4$}
	{$8$}
	{$12$}
	{$16$}
	\loigiai{Mô tả không gian mẫu ta có $$\Omega =\{SS;SN;NS;NN\}.$$
	Vậy không gian mẫu có $4$ phần tử.
	}
\end{ex}

\begin{ex}
	Nghiệm của phương trình $\tan x-1=0$ là
	\choice
	{$x=\dfrac{\pi}{6}+k \dfrac{\pi}{2}$, $k\in \mathbb{Z}$}
	{$x=\dfrac{3 \pi}{4}+k 2 \pi$, $k\in \mathbb{Z}$}
	{$x=-\dfrac{\pi}{4}+k \pi$, $k\in \mathbb{Z}$}
	{\True $x=\dfrac{\pi}{4}+k \pi$, $k\in \mathbb{Z}$}
	\loigiai{Ta có
	$$\tan x-1=0\Leftrightarrow \tan x=\tan \dfrac{\pi}{4}\Leftrightarrow x=\dfrac{\pi}{4}+k \pi,~ k\in \mathbb{Z}.$$
	}
\end{ex}

\begin{ex}
	Tập xác định của hàm số $y=\log _2 x$ là
	\choice
	{$[0 ;+\infty)$}
	{$(-\infty ;+\infty)$}
	{\True $(0 ;+\infty)$}
	{$[2 ;+\infty)$}
	\loigiai{
	Điều kiện $x>0$. \\Vậy  tập xác định của hàm số $y=\log _2 x$ là $\mathscr{D}=(0 ;+\infty)$.
	}
\end{ex}

\begin{ex}
	Trong mặt phẳng tọa độ $Oxy$, đường tròn tâm $I(3;-1)$ và bán kính $R=2$ có phương trình là
	\choice
	{$(x+3)^2+(y-1)^2=4$}
	{$(x-3)^2+(y-1)^2=4$}
	{\True $(x-3)^2+(y+1)^2=4$}
	{$(x+3)^2+(y+1)^2=4$}
	\loigiai{Đường tròn tâm $I(3 ;-1)$ và bán kính $R=2$ có phương trình là
	$$(x-3)^2+(y+1)^2=2^2\Leftrightarrow (x-3)^2+(y+1)^2=4.$$}
\end{ex}

\begin{ex}
	Cho hàm số $y=f(x)$ có bảng biến thiên trên nửa khoảng $[-5;7)$ như sau
	\begin{center}
		\begin{tikzpicture}
			\tkzTabInit[nocadre,lgt=1.2,espcl=2.5,deltacl=0.6]
			{$x$/0.6,$y'$/0.6,$y$/2}{$-5$,$1$,$7$}
			\tkzTabLine{,-,0,+,||}
			\tkzTabVar{+/$6$,-/$2$,+/$9$}
		\end{tikzpicture}
	\end{center}
	Mệnh đề nào dưới đây đúng?
	\choice
	{$\mathop {\min }\limits_{[ - 5;7)}  f(x)=6$}
	{\True $\mathop {\min }\limits_{[ - 5;7)} f(x)=2$}
	{$\mathop {\max }\limits_{[ - 5;7)} f(x)=9$}
	{$\mathop {\max }\limits_{[ - 5;7)} f(x)=6$}
	\loigiai{
	Từ bảng biến thiên ta có $\mathop {\min }\limits_{[ - 5;7)} f(x)=2$.}
\end{ex}

\end{document}
